\documentclass[11pt,a4paper]{article}

\usepackage[spanish]{babel}
\usepackage{graphicx}
\usepackage{mathtools}
\usepackage[margin=22mm]{geometry}
\usepackage{units}
\usepackage{fontspec}
%\setmainfont{Carlito}
\usepackage[hidelinks]{hyperref}
\usepackage{float}
\usepackage{cases}
\usepackage{wrapfig}
\usepackage{tikz}

%Libreria de programación con Matlab
\usepackage{listings}
\usepackage[framed,numbered,autolinebreaks,useliterate]{mcode}

%Formateo de encabezado y pie con definición de color unav
\definecolor{unavred}{HTML}{972d34}
\usepackage{fancyhdr}
\usepackage{lastpage}
\renewcommand{\headrule}{
	\vspace{-8pt}{\hrulefill}}
\pagestyle{fancy}
\fancyhf{}
\fancyhead[L]{\footnotesize\color{unavred} Título corto de la memoria}
\fancyhead[R]{\footnotesize\color{unavred} Carlos Mart\'inez de la Fuente}
\fancyfoot[R]{\color{unavred}$\nicefrac{\mathbf{\thepage}}{\mathbf{\pageref*{LastPage}}}$}

%Formateo de las secciones
\usepackage{titlesec}
\titleformat{\section}
{\color{unavred}\Large\normalfont\bfseries}
{\thesection}{1em}{\MakeUppercase}
%\newcommand{\addulc}[1]{\underline{\MakeUppercase{#1}}:}
\titleformat{\subsection}
{\large\normalfont\bfseries\color{unavred}}
{\thesubsection}{1em}{}

%Formateo de los caption
\usepackage[format=plain,labelfont={color=unavred,bf,it},textfont=it]{caption}
\usepackage[format=plain,labelfont={color=unavred,bf,it},textfont=it]{subcaption}

\usepackage{chngcntr}
\counterwithin*{figure}{section}
\counterwithin*{figure}{subsection}
\counterwithin*{figure}{subsubsection}
\renewcommand{\thefigure}{%
	\ifnum\value{subsection}=0
	\thesection.\arabic{figure}%
	\else
	\ifnum\value{subsubsection}=0
	\thesubsection.\arabic{figure}%
	\else
	\thesubsubsection.\arabic{figure}%
	\fi
	\fi
}

\setlength\parindent{0pt}

%Bibliografía
\usepackage[backend=biber,style=numeric,maxbibnames=3,sorting=none]{biblatex}
\addbibresource{biblio.bib}

\begin{document}
	\begin{titlepage}
		\begin{center}
			\vspace*{1cm}
			
			\Huge
			\textbf{Título de la memoria como alumno interno}
			
			\vspace{0.5cm}
			\LARGE
			Aplicación de la herramienta PhysiBoSS para la creación de gemelos digitales.
			
			\vspace{1.5cm}
			
			\textbf{Carlos Mart\'inez de la Fuente}
			
			\vfill
			
			
			
			\vspace{0.8cm}
			
			\includegraphics[width=0.4\textwidth]{Sello tecnun negro.jpg}
			
			\vspace{0.8cm}
			\Large
			Tecnun \--- Universidad de Navarra\\
			Alumno Interno\\
			Departamento de Biología Computacional\\
			\today
			
		\end{center}
	\end{titlepage}
	\tableofcontents
	\newpage
	\section{Introducción}
	En esta memoria se recoge lo aprendido y las experiencias adquiridas como alumno interno en el Departamento de Biología Computacional bajo la tutela del profesor D. Francisco Javier Planes.
	
	\subsection{¿Que es un gemelo digital?}
	Con la capacidad computacional de la que disponemos en el presente es posible generar y simular um sistema complejo de respuestas químicas y estructuras que nos permita ver la evolución dinámica de dicho sistema. Esto es fundamentalmente en lo que consiste la técnica del gemelo digital.
	\\Un ejemplo de esto lo tenemos en el trabajo de \textit{Ponce-de-Leon et al.}\supercite{physiboss2}, donde se simula la interacción de un sistema de células tumorales con un producto antitumoral.
	\\En este artículo\supercite{physiboss2} también se emplea y enseña la herramienta \textit{PhysiBoSS}\supercite{gitrepo} que será también el foco de este estudio como alumno interno.
	
	\subsection{¿Que es \textit{PhysiBoSS}?}
	\textit{PhysiBoSS} es la herramienta que utilizaremos. Se puede obtener desde el respositorio de Github: \url{https://github.com/PhysiBoSS/PhysiBoSS}.
	\\Esta herramienta busca fusionar las funciones de las herramientas de  \textit{PhysiCell}\supercite{physicell} y \textit{MaBoSS}\supercite{maboss} (de donde obtiene su nombre) de tal forma que no dependa de las anteriores, evitando que al actualizar cualquiera de las dos ocasione algún tipo de incompatibilidad.
	
	\subsubsection{Modelos Booleanos con \textit{MaBoSS}}
	\begin{wrapfigure}{hr}{0.4\textwidth}
		\vspace*{-12pt}
		\centering
		\scalebox{1.2}{
		\begin{tikzpicture}
			\begin{scope}[every node/.style={shape=circle,thick,draw=lightgray,fill=lightgray},text=unavred]
				\node (A) at (0,0) {A};
				\node (B) at (2,0) {B};
				\node (C) at (1,2) {C};
			\end{scope}
			
			\begin{scope}[every edge/.style={draw=unavred,very thick}]
				\path [-latex] (A) edge (B);
				\path [-latex] (B) edge (C);
				\path [-|] (C) edge (A);
			\end{scope}
		\end{tikzpicture}
	}
		\caption{Ejemplo de un sistema Booleano en el que A produce B, que produce C. C inhibe A.}
		\label{fig:bool_ejemplo}
	\end{wrapfigure}
	Como indican en su artículo \textit{Stoll et al.}\supercite{maboss}, "MaBoSS es un software que permite simular poblaciones de células y modelar de forma estocástica los mecanismos intracelulares".
	\\Un sistema Booleano (como el que se ve en la \hyperref[fig:bool_ejemplo]{Fig. \ref{fig:bool_ejemplo}}) consiste en una red de condiciones que son verdaderas o falsas (\textit{True} o \textit{False}) que implican o inhiben otra condición.
	\\En el ámbito celular, un modelo Booleano nos permite describir las relaciones que existen dentro de la célula a través de los diversos mensajeros químicos.
	\\\textit{MaBoSS} se apoya en esto para sus simulaciones dinámicas y estocásticas de sistemas.
	
	\subsubsection{Simulación de sistemas pluricelulares con \textit{PhysiCell}}
	

	\newpage
	\printbibliography[title=Bibliografía consultada,heading=bibintoc]
\end{document}
